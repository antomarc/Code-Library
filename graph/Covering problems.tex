\subsection{Covering Problems}
最大团以及相关知识\\
\\
独立集: 独立集是指图的顶点集的一个子集,该子集的导出子图的点互不相邻.如果一个独立集不是任何一个独立集的子集, 那么称这个独立集是一个极大独立集.一个图中包含顶点数目最多的独立集称为最大独立集。最大独立集一定是极大独立集,但是极大独立集不一定是最大的独立集。\\
\\
支配集: 与独立集相对应的就是支配集,支配集也是图顶点集的一个子集,设S是图G的一个支配集,则对于图中的任意一个顶点u,要么属于集合s, 要么与s中的顶点相邻。在s中除去任何元素后s不再是支配集,则支配集s是极小支配集。称G的所有支配集中顶点个数最少的支配集为最小支配集,最小支配集中的顶点个数成为支配数。\\
\\
最小点(对边)的覆盖: 最小点的覆盖也是图的顶点集的一个子集,如果我们选中一个点,则称这个点将以他为端点的所有边都覆盖了。将图中所有的边都覆盖所用顶点数最少,这个集合就是最小的点的覆盖。\\
\\
最大团: 图G的顶点的子集,设D是最大团,则D中任意两点相邻。若u,v是最大团,则u,v有边相连,其补图u,v没有边相连,所以图G的最大团=其补图的最大独立集。给定无向图G = (V;E),如果U属于V,并且对于任意u,v包含于U 有< u; v >包含于E,则称U是G的完全子图,G的完全子图U是G的团,当且仅当U不包含在G的更大的完全子图中,G的最大团是指G中所含顶点数目最多的团。如果U属于V,并且对于任意u; v包含于U有< u; v >不包含于E,则称U是G的空子图,G的空子图U是G的独立集,当且仅当U不包含在G的更大的独立集,G的最大团是指G中所含顶点数目最多的独立集。\\
\\
性质: \\
最大独立集+最小覆盖集= V\\
最大团=补图的最大独立集\\
最小覆盖集=最大匹配\\
\\
minimum cover:\\
edge cover vertex bipartite graph = maximum cardinality bipartite matching\\
找完最大二分匹配後,有三種情況要分別處理:\\
甲、 X 側未匹配點的交錯樹們。\\
乙、 Y 側未匹配點的交錯樹們。\\
丙、層層疊疊的交錯環們(包含單獨的匹配邊)。\\
這三個情況互不干涉。用 Graph Traversal 建立甲、乙的交錯樹們,剩下部分就是丙。\\
要找點覆蓋,甲、乙是取盡奇數距離的點,丙是取盡偶數距離的點、或者是取盡奇數距離的點,每塊連通分量可以各自為政。另外,小心處理的話,是可以印出字典順序最小的點覆蓋的。\\
已經有最大匹配時,求點覆蓋的時間複雜度等同於一次 Graph Traversal 的時間。\\
\\
vertex cover edge\\
\\
edge cover vertex\\
首先在圖上求得一個 Maximum Matching 之後,對於那些單身的點,都由匹配點連過去。如此便形成了 Minimum Edge Cover 。\\
\\
edge cover edge \\
\\
path cover vertex\\
general graph: NP-H\\
tree: DP\\
DAG: 将每个节点拆分为入点和出点,ans=节点数-匹配数\\
\\
path cover edge\\
minimize the count of euler path ( greedy is ok? )\\
dg[i]表示每个点的id-od,$ans=\sum dg[i], \forall dg[i]>0$\\
\\
cycle cover vertex\\
general: NP-H\\
weighted: do like path cover vertex, with KM algorithm\\
\\
cycle cover edge\\
NP-H\\
