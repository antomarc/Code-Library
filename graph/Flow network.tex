\subsection{Flow Network}
\subsubsection{Maximum weighted closure of a graph}
所有由这个子图中的点出发的边都指向这个子图,那么这个子图为原图的一个closure(闭合子图)\\
\\
每个节点向其所有依赖节点连边,容量inf\\
源点向所有正权值节点连边,容量为该权值\\
所有负权值节点向汇点连边,容量为该权值绝对值\\
以上均为有向边\\
最大权为sum\{正权值\}-\{新图的最小割\}\\
残量图中所有由源点可达的点即为所选子图
\subsubsection{Eulerian circuit}
计入度和出度之差\\
无向边任意定向\\
出入度之差为奇数则无解\\
然后构图:\\
原图有向边不变,容量1 // 好像需要在新图中忽略有向边?\\
无向边按之前认定方向,容量1\\
源点向所有度数为正的点连边,容量abs(度数/2)\\
所有度数为负的点向汇点连边,容量abs(度数/2)\\
两侧均满流则有解\\
相当于规约为可行流问题\\
注意连通性的trick\\
\\
终点到起点加一条有向边即可将path问题转为circuit问题
\subsubsection{Feasible flow problem}
由超级源点出发的边全部满流则有解\\
有源汇时,由汇点向源点连边,下界0上界inf即可转化为无源无汇上下界流\\
\\
对于每条边<a->b cap\{u,d\}>,建边<ss->b cap(u)>、<a->st cap(u)>、<a->b cap(d-u)>
\begin{itemize}
\item Maximum flow\\
//将流量还原至原图后,在残量网络上继续完成最大流\\
直接把source和sink设为原来的st,此时输出的最大流即是答案\\
不需要删除或者调整t->s弧
\item Minimum flow\\
建图时先不连汇点到源点的边,新图中完成最大流之后再连原汇至原源的边完成第二次最大流,此时t->s这条弧的流量即为最小流\\
判断可行流存在还是必须连原汇->原源的边之后查看满流\\
所以可以使用 跑流->加ts弧->跑流, 最后检查超级源点满流情况来一步搞定
\item tips\\
合并流量、减少边数来加速
\end{itemize}
\subsubsection{Minimum cost feasible flow problem}
TODO\\
看起来像是在上面那样跑费用流就行了……
\subsubsection{Minimum weighted vertex cover edge for bipartite graph}
for all vertex in X:\\
edge < s->x cap(weight(x)) >\\
for all vertex in Y:\\
edge < y->t cap(weight(y)) >\\
for original edges\\
edge < x->y cap(inf) >\\
\\
ans=\{maximum flow\}=\{minimum cut\}\\
残量网络中的所有简单割( (源点可达 \&\& 汇点不可达) || (源点不可达 \&\& 汇点可达) )对应着解
\subsubsection{Maximum weighted vertex independent set for bipartite graph}
ans=Sum\{点权\}-value\{Minimum weighted vertex cover edge\}\\
解应该就是最小覆盖集的补图吧……
\subsubsection{方格取数}
refer: hdu 3820 golden eggs\\
取方格获得收益\\
当取了相邻方格时付出边的代价\\
\\
必取的方格到源/汇的边的容量inf\\
相邻方格之间的边的容量为\{代价\}*2\\
ans=sum\{方格收益\}-\{最大流\}
\subsubsection{Uniqueness of min-cut}
refer:关键边。有向边起点为s集,终点为t集\\
从源和汇分别能够到的点集是所有点时,最小割唯一\\
也就是每一条增广路径都仅有一条边满流\\
注意查看的是实际的网络,不是残量网络\\
\\
具体来说
\begin{lstlisting}[language=C++]
void rr(int now)
{
    done[now]=true;
    ++cnt;
    for(int i(edge[now]);i!=-1;i=nxt[i])
        if(cap[i] && !done[v])
            rr(v);
}

void dfs(int now)
{
    done[now]=true;
    ++cnt;
    for(int i(edge[now]);i!=-1;i=nxt[i])
        if(cap[i^1] && !done[v])
            dfs(v);
}

memset(done,0,sizeof done);
cnt=0;
rr(source);
dfs(sink);
puts(cnt==n?"UNIQUE":"AMBIGUOUS");
\end{lstlisting}
\subsubsection{Tips}
\begin{itemize}
\item 两点间可以不止有一种边,也可以不止有一条边,无论有向无向
\item 两点间容量inf则可以设法化简为一个点
\item 点权始终要转化为边权
\item 不参与决策的边权设为inf来排除掉
\item 贪心一个初始不合法情况,然后通过可行流调整
\begin{itemize}
\item 混合图欧拉回路存在性
\item 有向/无向图中国邮差问题(遍历所有边至少一次后回到原点)
\end{itemize}
\item 按时间拆点(时间层……?)
\end{itemize}
