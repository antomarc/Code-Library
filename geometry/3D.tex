\subsection{3D}
\begin{lstlisting}[language=C++]
struct pv
{
	double x,y,z;
	pv() {}
	pv(double xx,double yy,double zz):x(xx),y(yy),z(zz)	{}
	pv operator -(const pv& b)const
	{
		return pv(x-b.x,y-b.y,z-b.z);
	}
	pv operator *(const pv& b)const
	{
		return pv(y*b.z-z*b.y,z*b.x-x*b.z,x*b.y-y*b.x);
	}
	double operator &(const pv& b)const
	{
		return x*b.x+y*b.y+z*b.z;
	}
};

//模
double Norm(pv p)
{
	return sqrt(p&p);
}

//`计算cross product U x V`
point3 xmult(point3 u,point3 v)
{
    point3 ret;
    ret.x=u.y*v.z-v.y*u.z;
    ret.y=u.z*v.x-u.x*v.z;
    ret.z=u.x*v.y-u.y*v.x;
    return ret;
}
//`计算dot product U . V`
double dmult(point3 u,point3 v)
{
    return u.x*v.x+u.y*v.y+u.z*v.z;
}
//`矢量差 U - V`
point3 subt(point3 u,point3 v)
{
    point3 ret;
    ret.x=u.x-v.x;
    ret.y=u.y-v.y;
    ret.z=u.z-v.z;
    return ret;
}
//`取平面法向量`
point3 pvec(plane3 s)
{
    return xmult(subt(s.a,s.b),subt(s.b,s.c));
}
point3 pvec(point3 s1,point3 s2,point3 s3)
{
    return xmult(subt(s1,s2),subt(s2,s3));
}
//`两点距离,单参数取向量大小`
double distance(point3 p1,point3 p2)
{
    return sqrt((p1.x-p2.x)*(p1.x-p2.x)+(p1.y-p2.y)*(p1.y-p2.y)+(p1.z-p2.z)*(p1.z-p2.z));
}
//`向量大小`
double vlen(point3 p)
{
    return sqrt(p.x*p.x+p.y*p.y+p.z*p.z);
}

\end{lstlisting}

\subsubsection{Geographic}
Geographic coordinate system coversion witch Cartesian coordinate system:\\
$x=r\times\sin(\theta)\times\cos(\alpha)$\\
$y=r\times\sin(\theta)\times\sin(\alpha)$\\
$z=r\times\cos(\theta)$\\
\\
$r=\sqrt{x^2+y^2+z^2}$\\
$\alpha$=atan(y/x);\\
$\theta$=acos(z/r);\\
\\
$r\in [0,\infty)$\\
$\alpha \in [0,2\pi]$\\
$\theta \in [0,\pi]$\\
\\
$lat\in [-\frac{\pi}{2},\frac{\pi}{2}]$\\
$lng\in [-\pi,\pi]$
\begin{lstlisting}[language=C++]
pv getpv(double lat,double lng,double r)
{
	lat += pi/2;
	lng += pi;
	return
    pv(r*sin(lat)*cos(lng),r*sin(lat)*sin(lng),r*cos(lat));
}
\end{lstlisting}

Distance in the suface of ball:
\begin{lstlisting}[language=C++]
#include<cstdio>
#include<cmath>

#define MAXX 1111

char buf[MAXX];
const double r=6875.0/2,pi=acos(-1.0);
double a,b,c,x1,x2,y2,ans;

int main()
{
    double y1;
    while(gets(buf)!=NULL)
    {
        gets(buf);
        gets(buf);

        scanf("%lf^%lf'%lf\" %s\n",&a,&b,&c,buf);
        x1=a+b/60+c/3600;
        x1=x1*pi/180;
        if(buf[0]=='S')
            x1=-x1;

        scanf("%s",buf);
        scanf("%lf^%lf'%lf\" %s\n",&a,&b,&c,buf);
        y1=a+b/60+c/3600;
        y1=y1*pi/180;
        if(buf[0]=='W')
            y1=-y1;

        gets(buf);

        scanf("%lf^%lf'%lf\" %s\n",&a,&b,&c,buf);
        x2=a+b/60+c/3600;
        x2=x2*pi/180;
        if(buf[0]=='S')
            x2=-x2;

        scanf("%s",buf);
        scanf("%lf^%lf'%lf\" %s\n",&a,&b,&c,buf);
        y2=a+b/60+c/3600;
        y2=y2*pi/180;
        if(buf[0]=='W')
            y2=-y2;

        ans=acos(cos(x1)*cos(x2)*cos(y1-y2)+sin(x1)*sin(x2))*r;
        printf("The distance to the iceberg: %.2lf miles.\n",ans);
        if(ans+0.005<100)
            puts("DANGER!");

        gets(buf);
    }
    return 0;
}
\end{lstlisting}

zju:
\subsubsection{Checks}
\begin{lstlisting}[language=C++]
//`判三点共线`
int dots_inline(point3 p1,point3 p2,point3 p3)
{
    return vlen(xmult(subt(p1,p2),subt(p2,p3)))<eps;
}
//`判四点共面`
int dots_onplane(point3 a,point3 b,point3 c,point3 d)
{
    return zero(dmult(pvec(a,b,c),subt(d,a)));
}
//`判点是否在线段上,包括端点和共线`
int dot_online_in(point3 p,line3 l)
{
    return zero(vlen(xmult(subt(p,l.a),subt(p,l.b))))&&(l.a.x-p.x)*(l.b.x-p.x)<eps&&
        (l.a.y-p.y)*(l.b.y-p.y)<eps&&(l.a.z-p.z)*(l.b.z-p.z)<eps;
}
int dot_online_in(point3 p,point3 l1,point3 l2)
{
    return zero(vlen(xmult(subt(p,l1),subt(p,l2))))&&(l1.x-p.x)*(l2.x-p.x)<eps&&
        (l1.y-p.y)*(l2.y-p.y)<eps&&(l1.z-p.z)*(l2.z-p.z)<eps;
}
//`判点是否在线段上,不包括端点`
int dot_online_ex(point3 p,line3 l)
{
    return dot_online_in(p,l)&&(!zero(p.x-l.a.x)||!zero(p.y-l.a.y)||!zero(p.z-l.a.z))&&
        (!zero(p.x-l.b.x)||!zero(p.y-l.b.y)||!zero(p.z-l.b.z));
}
int dot_online_ex(point3 p,point3 l1,point3 l2)
{
    return dot_online_in(p,l1,l2)&&(!zero(p.x-l1.x)||!zero(p.y-l1.y)||!zero(p.z-l1.z))&&
        (!zero(p.x-l2.x)||!zero(p.y-l2.y)||!zero(p.z-l2.z));
}
//`判点是否在空间三角形上,包括边界,三点共线无意义`
int dot_inplane_in(point3 p,plane3 s)
{
    return zero(vlen(xmult(subt(s.a,s.b),subt(s.a,s.c)))-vlen(xmult(subt(p,s.a),subt(p,s.b)))-
            vlen(xmult(subt(p,s.b),subt(p,s.c)))-vlen(xmult(subt(p,s.c),subt(p,s.a))));
}
int dot_inplane_in(point3 p,point3 s1,point3 s2,point3 s3)
{
    return zero(vlen(xmult(subt(s1,s2),subt(s1,s3)))-vlen(xmult(subt(p,s1),subt(p,s2)))-
            vlen(xmult(subt(p,s2),subt(p,s3)))-vlen(xmult(subt(p,s3),subt(p,s1))));
}
//`判点是否在空间三角形上,不包括边界,三点共线无意义`
int dot_inplane_ex(point3 p,plane3 s)
{
    return dot_inplane_in(p,s)&&vlen(xmult(subt(p,s.a),subt(p,s.b)))>eps&&
        vlen(xmult(subt(p,s.b),subt(p,s.c)))>eps&&vlen(xmult(subt(p,s.c),subt(p,s.a)))>eps;
}
int dot_inplane_ex(point3 p,point3 s1,point3 s2,point3 s3)
{
    return dot_inplane_in(p,s1,s2,s3)&&vlen(xmult(subt(p,s1),subt(p,s2)))>eps&&
        vlen(xmult(subt(p,s2),subt(p,s3)))>eps&&vlen(xmult(subt(p,s3),subt(p,s1)))>eps;
}
//`判两点在线段同侧,点在线段上返回0,不共面无意义`
int same_side(point3 p1,point3 p2,line3 l)
{
    return dmult(xmult(subt(l.a,l.b),subt(p1,l.b)),xmult(subt(l.a,l.b),subt(p2,l.b)))>eps;
}
int same_side(point3 p1,point3 p2,point3 l1,point3 l2)
{
    return dmult(xmult(subt(l1,l2),subt(p1,l2)),xmult(subt(l1,l2),subt(p2,l2)))>eps;
}
//`判两点在线段异侧,点在线段上返回0,不共面无意义`
int opposite_side(point3 p1,point3 p2,line3 l)
{
    return dmult(xmult(subt(l.a,l.b),subt(p1,l.b)),xmult(subt(l.a,l.b),subt(p2,l.b)))<-eps;
}
int opposite_side(point3 p1,point3 p2,point3 l1,point3 l2)
{
    return dmult(xmult(subt(l1,l2),subt(p1,l2)),xmult(subt(l1,l2),subt(p2,l2)))<-eps;
}
//`判两点在平面同侧,点在平面上返回0`
int same_side(point3 p1,point3 p2,plane3 s)
{
    return dmult(pvec(s),subt(p1,s.a))*dmult(pvec(s),subt(p2,s.a))>eps;
}
int same_side(point3 p1,point3 p2,point3 s1,point3 s2,point3 s3)
{
    return dmult(pvec(s1,s2,s3),subt(p1,s1))*dmult(pvec(s1,s2,s3),subt(p2,s1))>eps;
}
//`判两点在平面异侧,点在平面上返回0`
int opposite_side(point3 p1,point3 p2,plane3 s)
{
    return dmult(pvec(s),subt(p1,s.a))*dmult(pvec(s),subt(p2,s.a))<-eps;
}
int opposite_side(point3 p1,point3 p2,point3 s1,point3 s2,point3 s3)
{
    return dmult(pvec(s1,s2,s3),subt(p1,s1))*dmult(pvec(s1,s2,s3),subt(p2,s1))<-eps;
}
//`判两直线平行`
int parallel(line3 u,line3 v)
{
    return vlen(xmult(subt(u.a,u.b),subt(v.a,v.b)))<eps;
}
int parallel(point3 u1,point3 u2,point3 v1,point3 v2)
{
    return vlen(xmult(subt(u1,u2),subt(v1,v2)))<eps;
}
//`判两平面平行`
int parallel(plane3 u,plane3 v)
{
    return vlen(xmult(pvec(u),pvec(v)))<eps;
}
int parallel(point3 u1,point3 u2,point3 u3,point3 v1,point3 v2,point3 v3)
{
    return vlen(xmult(pvec(u1,u2,u3),pvec(v1,v2,v3)))<eps;
}
//`判直线与平面平行`
int parallel(line3 l,plane3 s)
{
    return zero(dmult(subt(l.a,l.b),pvec(s)));
}
int parallel(point3 l1,point3 l2,point3 s1,point3 s2,point3 s3)
{
    return zero(dmult(subt(l1,l2),pvec(s1,s2,s3)));
}
//`判两直线垂直`
int perpendicular(line3 u,line3 v)
{
    return zero(dmult(subt(u.a,u.b),subt(v.a,v.b)));
}
int perpendicular(point3 u1,point3 u2,point3 v1,point3 v2)
{
    return zero(dmult(subt(u1,u2),subt(v1,v2)));
}
//`判两平面垂直`
int perpendicular(plane3 u,plane3 v)
{
    return zero(dmult(pvec(u),pvec(v)));
}
int perpendicular(point3 u1,point3 u2,point3 u3,point3 v1,point3 v2,point3 v3)
{
    return zero(dmult(pvec(u1,u2,u3),pvec(v1,v2,v3)));
}
//`判直线与平面平行`
int perpendicular(line3 l,plane3 s)
{
    return vlen(xmult(subt(l.a,l.b),pvec(s)))<eps;
}
int perpendicular(point3 l1,point3 l2,point3 s1,point3 s2,point3 s3)
{
    return vlen(xmult(subt(l1,l2),pvec(s1,s2,s3)))<eps;
}
//`判两线段相交,包括端点和部分重合`
int intersect_in(line3 u,line3 v)
{
    if (!dots_onplane(u.a,u.b,v.a,v.b))
        return 0;
    if (!dots_inline(u.a,u.b,v.a)||!dots_inline(u.a,u.b,v.b))
        return !same_side(u.a,u.b,v)&&!same_side(v.a,v.b,u);
    return dot_online_in(u.a,v)||dot_online_in(u.b,v)||dot_online_in(v.a,u)||dot_online_in(v.b,u);
}
int intersect_in(point3 u1,point3 u2,point3 v1,point3 v2)
{
    if (!dots_onplane(u1,u2,v1,v2))
        return 0;
    if (!dots_inline(u1,u2,v1)||!dots_inline(u1,u2,v2))
        return !same_side(u1,u2,v1,v2)&&!same_side(v1,v2,u1,u2);
    return
        dot_online_in(u1,v1,v2)||dot_online_in(u2,v1,v2)||dot_online_in(v1,u1,u2)||dot_online_in(v2,u1,u
                2);
}
//`判两线段相交,不包括端点和部分重合`
int intersect_ex(line3 u,line3 v)
{
    return dots_onplane(u.a,u.b,v.a,v.b)&&opposite_side(u.a,u.b,v)&&opposite_side(v.a,v.b,u);
}
int intersect_ex(point3 u1,point3 u2,point3 v1,point3 v2)
{
    return
        dots_onplane(u1,u2,v1,v2)&&opposite_side(u1,u2,v1,v2)&&opposite_side(v1,v2,u1,u2);
}
//`判线段与空间三角形相交,包括交于边界和(部分)包含`
int intersect_in(line3 l,plane3 s)
{
    return !same_side(l.a,l.b,s)&&!same_side(s.a,s.b,l.a,l.b,s.c)&&
        !same_side(s.b,s.c,l.a,l.b,s.a)&&!same_side(s.c,s.a,l.a,l.b,s.b);
}
int intersect_in(point3 l1,point3 l2,point3 s1,point3 s2,point3 s3)
{
    return !same_side(l1,l2,s1,s2,s3)&&!same_side(s1,s2,l1,l2,s3)&&
        !same_side(s2,s3,l1,l2,s1)&&!same_side(s3,s1,l1,l2,s2);
}
//`判线段与空间三角形相交,不包括交于边界和(部分)包含`
int intersect_ex(line3 l,plane3 s)
{
    return opposite_side(l.a,l.b,s)&&opposite_side(s.a,s.b,l.a,l.b,s.c)&&
        opposite_side(s.b,s.c,l.a,l.b,s.a)&&opposite_side(s.c,s.a,l.a,l.b,s.b);
}
int intersect_ex(point3 l1,point3 l2,point3 s1,point3 s2,point3 s3)
{
    return opposite_side(l1,l2,s1,s2,s3)&&opposite_side(s1,s2,l1,l2,s3)&&
        opposite_side(s2,s3,l1,l2,s1)&&opposite_side(s3,s1,l1,l2,s2);
}

//mzry
inline bool ZERO(const double &a)
{
    return fabs(a)<eps;
}

inline bool ZERO(pv p)
{ 
    return (ZERO(p.x) && ZERO(p.y) && ZERO(p.z)); 
} 

//直线相交
bool LineIntersect(Line3D L1, Line3D L2) 
{ 
    pv s = L1.s-L1.e; 
    pv e = L2.s-L2.e; 
    pv p  = s*e; 
    if (ZERO(p)) 
        return false;    //是否平行 
    p = (L2.s-L1.e)*(L1.s-L1.e); 
    return ZERO(p&L2.e);         //是否共面 
} 

//线段相交
bool inter(pv a,pv b,pv c,pv d)
{
    pv ret = (a-b)*(c-d);
    pv t1 = (b-a)*(c-a);
    pv t2 = (b-a)*(d-a);
    pv t3 = (d-c)*(a-c);
    pv t4 = (d-c)*(b-c);
    return sgn(t1&ret)*sgn(t2&ret) < 0 && sgn(t3&ret)*sgn(t4&ret) < 0;
}

//点在直线上
bool OnLine(pv p, Line3D L)
{ 
    return ZERO((p-L.s)*(L.e-L.s)); 
} 

//点在线段上
bool OnSeg(pv p, Line3D L)
{ 
    return (ZERO((L.s-p)*(L.e-p)) && EQ(Norm(p-L.s)+Norm(p-L.e),Norm(L.e-L.s))); 
} 

//点到直线距离
double Distance(pv p, Line3D L)
{ 
    return (Norm((p-L.s)*(L.e-L.s))/Norm(L.e-L.s)); 
} 
\end{lstlisting}
\subsubsection{Intersection}
\begin{lstlisting}[language=C++]
//`计算两直线交点,注意事先判断直线是否共面和平行!`
//`线段交点请另外判线段相交(同时还是要判断是否平行!)`
point3 intersection(line3 u,line3 v)
{
    point3 ret=u.a;
    double t=((u.a.x-v.a.x)*(v.a.y-v.b.y)-(u.a.y-v.a.y)*(v.a.x-v.b.x))
        /((u.a.x-u.b.x)*(v.a.y-v.b.y)-(u.a.y-u.b.y)*(v.a.x-v.b.x));
    ret.x+=(u.b.x-u.a.x)*t;
    ret.y+=(u.b.y-u.a.y)*t;
    ret.z+=(u.b.z-u.a.z)*t;
    return ret;
}
point3 intersection(point3 u1,point3 u2,point3 v1,point3 v2)
{
    point3 ret=u1;
    double t=((u1.x-v1.x)*(v1.y-v2.y)-(u1.y-v1.y)*(v1.x-v2.x))
        /((u1.x-u2.x)*(v1.y-v2.y)-(u1.y-u2.y)*(v1.x-v2.x));
    ret.x+=(u2.x-u1.x)*t;
    ret.y+=(u2.y-u1.y)*t;
    ret.z+=(u2.z-u1.z)*t;
    return ret;
}
//`计算直线与平面交点,注意事先判断是否平行,并保证三点不共线!`
//`线段和空间三角形交点请另外判断`
point3 intersection(line3 l,plane3 s)
{
    point3 ret=pvec(s);
    double t=(ret.x*(s.a.x-l.a.x)+ret.y*(s.a.y-l.a.y)+ret.z*(s.a.z-l.a.z))/
        (ret.x*(l.b.x-l.a.x)+ret.y*(l.b.y-l.a.y)+ret.z*(l.b.z-l.a.z));
    ret.x=l.a.x+(l.b.x-l.a.x)*t;
    ret.y=l.a.y+(l.b.y-l.a.y)*t;
    ret.z=l.a.z+(l.b.z-l.a.z)*t;
    return ret;
}
point3 intersection(point3 l1,point3 l2,point3 s1,point3 s2,point3 s3)
{
    point3 ret=pvec(s1,s2,s3);
    double t=(ret.x*(s1.x-l1.x)+ret.y*(s1.y-l1.y)+ret.z*(s1.z-l1.z))/
        (ret.x*(l2.x-l1.x)+ret.y*(l2.y-l1.y)+ret.z*(l2.z-l1.z));
    ret.x=l1.x+(l2.x-l1.x)*t;
    ret.y=l1.y+(l2.y-l1.y)*t;
    ret.z=l1.z+(l2.z-l1.z)*t;
    return ret;
}
//`计算两平面交线,注意事先判断是否平行,并保证三点不共线!`
line3 intersection(plane3 u,plane3 v)
{
    line3 ret;
    ret.a=parallel(v.a,v.b,u.a,u.b,u.c)?intersection(v.b,v.c,u.a,u.b,u.c):intersection(v.a,v.b,u.a,u.b,u.
            c);
    ret.b=parallel(v.c,v.a,u.a,u.b,u.c)?intersection(v.b,v.c,u.a,u.b,u.c):intersection(v.c,v.a,u.a,u.b,u.
            c);
    return ret;
}
line3 intersection(point3 u1,point3 u2,point3 u3,point3 v1,point3 v2,point3 v3)
{
    line3 ret;
    ret.a=parallel(v1,v2,u1,u2,u3)?intersection(v2,v3,u1,u2,u3):intersection(v1,v2,u1,u2,u3);
    ret.b=parallel(v3,v1,u1,u2,u3)?intersection(v2,v3,u1,u2,u3):intersection(v3,v1,u1,u2,u3);
    return ret;
}
\end{lstlisting}
\subsubsection{Distance}
\begin{lstlisting}[language=C++]
//`点到直线距离`
double ptoline(point3 p,line3 l)
{
    return vlen(xmult(subt(p,l.a),subt(l.b,l.a)))/distance(l.a,l.b);
}
double ptoline(point3 p,point3 l1,point3 l2)
{
    return vlen(xmult(subt(p,l1),subt(l2,l1)))/distance(l1,l2);
}
//`点到平面距离`
double ptoplane(point3 p,plane3 s)
{
    return fabs(dmult(pvec(s),subt(p,s.a)))/vlen(pvec(s));
}
double ptoplane(point3 p,point3 s1,point3 s2,point3 s3)
{
    return fabs(dmult(pvec(s1,s2,s3),subt(p,s1)))/vlen(pvec(s1,s2,s3));
}
//`直线到直线距离`
double linetoline(line3 u,line3 v)
{
    point3 n=xmult(subt(u.a,u.b),subt(v.a,v.b));
    return fabs(dmult(subt(u.a,v.a),n))/vlen(n);
}
double linetoline(point3 u1,point3 u2,point3 v1,point3 v2)
{
    point3 n=xmult(subt(u1,u2),subt(v1,v2));
    return fabs(dmult(subt(u1,v1),n))/vlen(n);
}
\end{lstlisting}
\subsubsection{Angle}
\begin{lstlisting}[language=C++]
//`两直线夹角cos 值`
double angle_cos(line3 u,line3 v)
{
    return dmult(subt(u.a,u.b),subt(v.a,v.b))/vlen(subt(u.a,u.b))/vlen(subt(v.a,v.b));
}
double angle_cos(point3 u1,point3 u2,point3 v1,point3 v2)
{
    return dmult(subt(u1,u2),subt(v1,v2))/vlen(subt(u1,u2))/vlen(subt(v1,v2));
}
//`两平面夹角cos 值`
double angle_cos(plane3 u,plane3 v)
{
    return dmult(pvec(u),pvec(v))/vlen(pvec(u))/vlen(pvec(v));
}
double angle_cos(point3 u1,point3 u2,point3 u3,point3 v1,point3 v2,point3 v3)
{
    return dmult(pvec(u1,u2,u3),pvec(v1,v2,v3))/vlen(pvec(u1,u2,u3))/vlen(pvec(v1,v2,v3));
}
//`直线平面夹角sin 值`
double angle_sin(line3 l,plane3 s)
{
    return dmult(subt(l.a,l.b),pvec(s))/vlen(subt(l.a,l.b))/vlen(pvec(s));
}
double angle_sin(point3 l1,point3 l2,point3 s1,point3 s2,point3 s3)
{
    return dmult(subt(l1,l2),pvec(s1,s2,s3))/vlen(subt(l1,l2))/vlen(pvec(s1,s2,s3));
}
\end{lstlisting}
