\subsection{Combinatorics}

\subsubsection{Subfactorial}
$!n=$number of permutations of n elements with no fixed points\\
    \\
from !0:\\
1, 0, 1, 2, 9, 44, 265, 1854, 14833, 133496, 1334961, 14684570\\
$!n=(n-1)(!(n-1)+!(n-2))$\\
PS:$n!=(n-1)((n-1)!+(n-2)!)$\\
$!n=n\times n!+(-1)^n$\\
\\
Rencontres numbers:\\
$D_{n,k}$ is the number of permutations of \{ 1, ..., n \} that have exactly k fixed points.\\
$D_{n,0}=!n$\\
$D_{n,k}=\binom{n}{k}\times !(n-k)$\\

\subsubsection{M\'{e}nage numbers}
M\'{e}nage numbers:\\
number of permutations s of [0, ..., n-1] such that.\\
$ \forall i, s(i) \neq i$ and $s(i) \not\equiv i+1 \pmod n$.\\
\\
from A(0):\\
1, 0, 0, 1, 2, 13, 80, 579, 4738, 43387, 439792, 4890741\\

$A_n=\sum\limits_{k=0}^n (-1)^k \frac{2n}{2n-k} {2n-k\choose k} (n-k)!$\\
$A_n = n A_{n-1} + \frac{n}{n-2} A_{n-2} + \frac{4(-1)^{n-1}}{n-2}$\\
$A_n = n A_{n-1} + 2 A_{n-2} - (n-4)A_{n-3} - A_{n-4}$

\subsubsection{Multiset}
Permutation:\\
MultiSet S=\{1 m,4 s,4 i,2 p\}\\
$P(S)=\frac{(1+4+4+2)!}{1!4!4!2!}$\\
\\
Combination:\\
MultiSet S=$\{\infty a1,\infty a2,... \infty ak\}$\\
$\binom{S}{r}=\frac{(r+k-1)!}{r!(k-1)!}=\binom{r+k-1}{r}$\\
\\
if(r>min\{count(element[i])\})\\
    you have to resolve this problem with inclusion-exclusion principle.\\
\\
MS T=\{3 a,4 b,5 c\}\\
MS $T_*=\{\infty a,\infty b,\infty c\}$\\
$A1=\{\binom{T_*}{10}|count(a)>3\} // \binom{8}{6}$\\
$A2=\{\binom{T_*}{10}|count(b)>4\} // \binom{7}{5}$\\
$A3=\{\binom{T_*}{10}|count(c)>5\} // \binom{6}{4}$\\
\\
$\binom{T}{10}=\binom{T_*}{10}-(|A_1|+|A_2|+|A_3|)+(|A_1 \cap A_2|+|A_1 \cap A_3|+|A_2 \cap A_3|)-|A_1 \cap A_2 \cap A_3|$\\
ans=C(10,12)-(C(6,8)+C(5,7)+C(4,6))+(C(1,3)+C(0,2)+0)-0=6\\

\subsubsection{Distributing Balls into Boxes}
Distributing m Balls into n Boxes.\\
\begin{tabular}{|l|l|l|l|}
\hline
balls & boxes & empty & counts\\
\hline
diff & diff & empty & $n^m$\\
\hline
diff & diff & full & $n!\times S(m,n)=\sum\limits_{i=0}^{n}{(-1)^n\binom{n}{i}(n-i)^m}$(inclusion-exclusion principle)\\
\hline
diff & same & empty & $\sum\limits_{k=1}^{min\{n,m\}}s(m,k)=\frac{1}{n!}\sum\limits_{k=1}^{min\{n,m\}}\sum\limits_{i=0}^k{(-1)^i\binom{k}{i}(k-i)^m}$\\
\hline
diff & same & full & S(m,n) (Stirling numbers of the second kind)\\
\hline
same & diff & empty & $\binom{n+m-1}{n-1}$\\
\hline
same & diff & full & $\binom{m-1}{n-1}$\\
\hline
same & same & empty & 
\begin{tabular}{l}
dp[0][0..n]=dp[1..m][1]=1;\\
if(m$\geq$n)\\ dp[m][n]=dp[m][n-1]+dp[m-n][n];\\
                                              else\\dp[m][n]=dp[m][n-1];\\
\end{tabular}\\
\hline
same & same & full & g[m][n]=dp[m-n][n];\\
\hline
\end{tabular}

\subsubsection{Combinatorial Game Theory}
Wythoff's game:
        \begin{itemize}
\item There are two piles of counters.
\item Players take turns removing counters (at least 1 counter) from one or both piles; in the latter case, the numbers of counters removed from each pile must be equal.
\item The player who removes the last counter wins. 
    \end{itemize}
consider the counters of status as pair (a,b) ($a\leq b$)\\
\{first player loses\} $\iff a=\lfloor (b-a)\times \phi \rfloor, \phi=\frac{\sqrt{5}+1}{2}$\\
    \\
Fibonacci Nim:
        \begin{itemize}
\item There is one pile of n counters.
\item The first player may remove any positive number of counters, but not the whole pile.
\item Thereafter, each player may remove at most twice the number of counters his opponent took on the previous move.
\item The player who removes the last counter wins.
    \end{itemize}
\{first player wins\}$\iff n \not\in$\{Fibonacci number\}\\
        \\
poj 1740:
    \begin{itemize}
\item There are n piles of stones.
\item At each step of the game,the player choose a pile,remove at least one stones,then freely move stones from this pile to any other pile that still has stones. 
\item The player who removes the last counter wins.
    \end{itemize}
    \{first player lose\}$\iff$n is even \&\& $(a_1,a_2,...,a_k)(a_1\leq a_2\leq ...\leq a_{2k})$ satisfy $a_{2i-1}=a_{2i}\{\forall i \in [1,k]\}$\\
\\

Staircase Nim:
    \begin{itemize}
    \item A staircase of n steps contains coins on some of the steps.
    \item A move of staircase nim consists of moving any positive number of coins from any step j , to the next lower step, j − 1.
    \item Coins reaching the ground (step 0) are removed from play.
    \item The player who removes the last counter wins.
    \end{itemize}
    Even steps are unusefull.\\
    $SG=x_1\oplus x_3\oplus x_5...$\\
\\
Anti-SG:
\begin{itemize}
\item Everything is likes SG.
\item The player who removes the last counter loses.
\end{itemize}
\{first player wins\}$\iff$\\
        SGsum=0,\&\& \{all piles is 1\}\\
     SGsum$\neq$0,\&\& \{some piles ars larger than 1\}\\
     \\
Every-SG:
\begin{itemize}
\item Everything is likes SG.
\item For each turns, player have to move all of sub-games if the sub-game was not ended yet.
\end{itemize}
\{first player wins\}$\iff$ max(steps of all sub-games) is odd.\\
\\
Coin Game:
\begin{itemize}
\item Given a horizontal line of N coins with some coins showing heads and some tails.
\item Each turn, a player have to follow some rules, flip some coins. But the most right coin he fliped has to be fliped from head to tail.
\item The player who can not flip coin loses.
\end{itemize}
game\{THHTTH\} = game\{TH\}$\oplus$game\{TTH\}$\oplus$game\{TTTTTH\}\\
        \\
Tree Game:
\begin{itemize}
\item There is a rooted tree.
\item Each turn, a player has to remove a edge from the tree. The parts can not connect with root with also are removed.
\item The player who removes the last edge wins.
\end{itemize}
$\forall node(x),\\
        SG(x)=(SG(i_1)+1)\oplus (SG(i_2)+1)\oplus ... (\forall$ i are childnodes of x)\\
        \\
Undirectional Graph Game:
\begin{itemize}
\item There is a rooted undirectional graph.
\item Other rules are likes Tree Game.
\end{itemize}
Odd Circle's SG value is 1.\\
Even Circel's SG value is 0.\\
turn the graph to a tree.\\

\subsubsection{Catalan number}
from $C_0$\\
1, 1, 2, 5, 14, 42, 132, 429, 1430, 4862, 16796, 58786, 208012, 742900, 2674440, 9694845, 35357670, 129644790, 477638700, 1767263190, 6564120420\\
$C_0=1$\\
$C_{n+1}=\sum\limits_{i=0}^nC_iC_{n-i}$\\
$C_{n+1}=\frac{2(2n+1)}{n+1}C_n$\\
\\
$C_n=\binom{2n}{n}-\binom{2n}{n+1}=\frac{1}{n+1}\binom{2n}{n}=\frac{(2n)!}{(n+1)!n!}$\\
$C_n \sim \frac{4^n}{n^{3/2}\sqrt{\pi}}$\\
Applications:\\
\begin{enumerate}
\item $C_n$ counts the number of expressions containing n pairs of parentheses which are correctly matched.
\item $C_n$ is the number of full binary trees with n + 1 leaves.
\item $C_n$ is the number of non-isomorphic ordered trees with n+1 vertices. (An ordered tree is a rooted tree in which the children of each vertex are given a fixed left-to-right order.)
\item $C_n$ is the number of monotonic paths along the edges of a grid with n × n square cells, which do not pass above the diagonal.($x\leq y$ for $C_n$, $x<y$ for $C_n-1$)
\begin{enumerate}
\item for the rectangle (p,q),($x<y$), $ans=\binom{p+q-1}{p}-\binom{p+q-1}{p-1}=\frac{q-p}{q+p}\binom{p+q}{q}$
\item for the rectangle (p,q),($x\leq y$),$ans=\binom{p+q}{p}-\binom{p+q}{p-1}=\frac{q-p+1}{q+1}\binom{p+q}{q}$
\end{enumerate}
\item $C_n$ is the number of different ways a convex polygon with n + 2 sides can be cut into triangles by connecting vertices with straight lines.
\item $C_n$ is the number of permutations of \{1, ..., n\} that avoid the pattern 123.
\item $C_n$ is the number of ways to tile a stairstep shape of height n with n rectangles.
\end{enumerate}

\subsubsection{Stirling number}
First kind:\\
Stirling numbers of the first kind is signed.\\
The unsigned Stirling numbers of the first kind are denoted by s(n,k).\\
s(4,2)=11\\
s(n,k) count the number of permutations of n elements with k disjoint cycles.\\
s(n,0)=s(1,1)=1\\
s(n+1,k)=s(n,k-1)+n s(n,k)\\
\\
Second kind:\\
S(n,k) count the number of ways to partition a set of n labelled objects into k nonempty unlabelled subsets.\\
S(4,2)=7\\
S(n,n)=S(n,1)=1\\
S(n,k)=S(n-1,k-1)+k S(n-1,k)\\
$S(n,n-1)=\binom{n}{2}=\frac{n(n-1)}{2}$\\
$S(n,2)=2^{n-1}-1$\\

\subsubsection{Delannoy number}
Delannoy number D describes the number of paths from (0, 0) to (m, n), using only single steps north, northeast, or east.\\
D(0,0)=1\\
D(m,n)=D(m-1,n)+D(m-1,n-1)+D(m,n-1)\\
\\
central Delannoy numbers D(n) = D(n,n)\\
D(n) from 0:\\
1, 3, 13, 63, 321, 1683, 8989, 48639, 265729\\
$nD(n)=3(2n-1)D(n-1)-(n-1)D(n-2)$\\

\subsubsection{Schr\"{o}der number}
Large:\\
Describes the number of paths from (0, 0) to (m, n), using only single steps north, northeast, or east, for all (x,y), ($x\leq y$).\\
for(n==m),from 0:\\
1, 2, 6, 22, 90, 394, 1806, 8558, 41586, 206098\\
$S(n)=S(n-1)+\sum\limits_{k=0}^{n-1}S(k)S(n-1-k)$\\
\\

Little: (aka. super-Catalan numbers, Hipparchus numbers)\\
\begin{enumerate}
\item the number of different trees with n leaves and with all internal vertices having two or more children.
\item the number of ways of inserting brackets into a sequence.
\item the number of ways of dissecting a convex polygon into smaller polygons by inserting diagonals.
\end{enumerate}
from 0:\\
1, 1, 3, 11, 45, 197, 903, 4279, 20793, 103049\\
s(n)=S(n)/2\\
s(0)=s(1)=1\\
ns(n)=(6n-9)s(n-1)-(n-3)s(n-2)\\
$a(n+1)=-a(n)+2\sum\limits_{k=1}^na(k)\times a(n+1-k)$\\
$a(n+1)=\sum\limits_{k=0}^{(n-1)/2} 2^k\times 3^{n-1-2k}\binom{n-1}{2k}$\\

\subsubsection{Bell number}
Number of partitions of a set of n labeled elements.\\
from 0:\\
1, 1, 2, 5, 15, 52, 203, 877, 4140, 21147, 115975\\
$B_{n+1}=\sum\limits_{k=0}^{n}{{n \choose k}B_k}$\\
$B_{p+n}\equiv B_n+B_{n+1}\pmod p$ (p for prime)\\
$B_{p^m+n}\equiv mB_n+B_{n+1}\pmod p$ (p for prime)\\
$B_n=\sum\limits_{k=1}^{n}S(n,k)$(S for Stirling second kind)\\

\subsubsection{Eulerian number}
First kind:\\
the number of permutations of the numbers 1 to n in which exactly m elements are greater than the previous element\\
A(n,0)=1\\
A(n,m)=(n-m)A(n-1,m-1)+(m+1)A(n-1,m)\\
A(n,m)=(n-m+1)A(n-1,m-1)+mA(n-1,m)\\
A(n,m)=A(n,n-1-m)\\
\\
Second kind:\\
count the permutations of the multiset \{1,1,2,2,...,n,n\} with k ascents with the restriction that for all m\\
T(n,0)=1\\
T(n,m)=(2n-m-1)T(n-1,m-1)+(m+1)T(n-1,m)\\
\\

\subsubsection{Motzkin number}
\begin{enumerate}
\item  the number of different ways of drawing non-intersecting chords on a circle between n points
\item Number of sequences of length n-1 consisting of positive integers such that the opening and ending elements are 1 or 2 and the absolute difference between any 2 consecutive elements is 0 or 1
\item paths from (0,0) to (n,0) in an n X n grid using only steps U = (1,1), F = (1,0) and D = (1,-1)
\end{enumerate}
from 0:\\
1, 1, 2, 4, 9, 21, 51, 127, 323, 835, 2188, 5798, 15511, 41835, 113634, 310572, 853467\\
$M_{n+1}=M_n+\sum\limits_{i=0}^{n-1}M_iM_{n-1-i}=\frac{2n+3}{n+3}M_n+\frac{3n}{n+3}M_{n-1}$\\
$M_n=\sum\limits_{k=0}^{\lfloor n/2\rfloor}\binom{n}{2k}C_k$(C for catalan)\\

\subsubsection{Narayana number}
\begin{enumerate}
\item the number of expressions containing n pairs of brackets which are correctly matched and which contain k pairs of ().
\item the number of paths from (0, 0) to (2n, 0), with steps only northeast and southeast, not straying below the x-axis, with k peaks.
\end{enumerate}
N(n,0)=0\\
$N(n,k)=\frac{1}{n}\binom{n}{k}\binom{n}{k-1}$\\
$N(n,k)=\frac{1}{k}\binom{n-1}{k-1}\binom{n}{k-1}$\\
$\sum\limits_{k=1}^nN(n,k)=C_n$(C for catalan)\\


\subsection{Number theory}
\subsubsection{Divisor Fuction}
$n=p_1^{a_1}\times p_2^{a_2}\times ... \times p_s^{a_s}$\\
sum of positive divisors function\\
$\sigma(n)=\prod\limits_{j=1}^s\frac{p_j^{a_j+1}-1}{p_j-1}$\\
number of postive diversors function\\
$\tau(n)=\prod\limits_{j=1}^s(a_j+1)$

\subsubsection{Reduced Residue System}
Euler's totient function:\\
\\
对正整数n,欧拉函数$\varphi$是小于或等于n的数中与n互质的数的数目,也就是对n的简化剩余系的大小。\\
 $\varphi$(2)=1(唯一和1互质的数就是1本身)。\\
若m,n互质,$\varphi (m\times n ) = \varphi (m ) \times \varphi (n )$。\\
对于n来说,所有这样的数的和为 $\frac{n\times\varphi (n )}{2}$ 。\\
$gcd(k,n)=d, k\in [1,n]$,这样的k有$\varphi(\frac{n}{d})个$\\\

\begin{lstlisting}[language=C++]
inline int phi(int n)
{
    static int i;
    static int re;
    re=n;
    for(i=0;prm[i]*prm[i]<=n;++i)
        if(n%prm[i]==0)
        {
            re-=re/prm[i];
            do
                n/=prm[i];
            while(n%prm[i]==0);
        }
    if(n!=1)
        re-=re/n;
    return re;
}

inline void Euler()
{
    static int i,j;
    phi[1]=1;
    for(i=2;i<MAXX;++i)
        if(!phi[i])
            for(j=i;j<MAXX;j+=i)
            {
                if(!phi[j])
                    phi[j]=j;
                phi[j]=phi[j]/i*(i-1);
            }
}
\end{lstlisting}
Multiplicative order:\\
\\
the multiplicative order of a modulo n is the smallest positive integer k with\\
     $a^k \equiv 1 \pmod n$ \\
\\
对m的简化剩余系中的所有x,ord(x)都一定是$\varphi$(m)的一个约数 (aka. Euler's totient theorem)\\
\\
求:\\
method 1、根据定义,对$\varphi$(m)分解素因子之后暴力寻找最小的一个$d\{d|\varphi(m)\}$,满足 $x^d \equiv 1 \pmod m$;\\
method 2、\\
\begin{lstlisting}[language=C++]
inline long long ord(long long x,long long m)
{
    static long long ans;
    static int i,j;
    ans=phi(m);
    for(i=0;i<fac.size();++i)
        for(j=0;j<fac[i].second && pow(x,ans/fac[i].first,m)==1ll;++j)
            ans/=fac[i].first;
    return ans;
}
\end{lstlisting}
Primitive root:\\
\\
若ord(x)==$\varphi$(m),则x为m的一个原根\\
因此只需检查所有 $x^d$ \{$d|\varphi(m)$\} 找到使 $x^d \equiv 1 \pmod m$ 的所有d,当且仅当这样的d只有一个,并且为$\varphi$(m)的时候,x是m的一个原根\\
\\
当且仅当m= 1,2,4,$p^n$,$2\times p^n$ \{p为奇质数,n为正整数\} 时,m存在原根 // 应该是指存在对于完全剩余系的原根……?\\
\\
当m存在原根时,原根数目为$\varphi$($\varphi$(m))\\
\\
求:\\
枚举每一个简化剩余系中的数i,若对于i的每一个质因子p[j],$i^{\frac{\varphi (m)}{p[j]}} \not\equiv 1 \pmod m$,那么i为m的一个原根。也就是说,ord(i)==$\varphi$(m)。\\
最小原根通常极小。\\
\\
Carmichael function:\\
\\
$\lambda$(n) is defined as the smallest positive integer m such that \\
    $a^m \equiv 1 \pmod n\{ \forall a!=1 \&\& gcd(a,n)==1 \}$\\
也就是简化剩余系(完全剩余系中存在乘法群中无法得到1的数)中所有x的 lcm\{ord(x)\}\\
\\
if n=$p[0]^{a[0]} \times p[1]^{a[1]} \times ... \times p[m-1]^{a[m-1]}$\\
    then $\lambda$(n)=lcm($\lambda$($p[0]^{a[0]}$),$\lambda$($p[1]^{a[1]}$),...,$\lambda$($p[m-1]^{a[m-1]}$));\\
\\
if n=$2^c \times p[0]^{a[0]} \times p[1]^{a[1]} \times ... \times p[m-1]^{a[m-1]}$\\
    then $\lambda$(n)=lcm($2^c$,$\varphi$($p[0]^{a[0]}$),$\varphi$($p[1]^{a[1]}$),...,$\varphi$($p[m-1]^{a[m-1]}$));\\
    { c=0 if a<2; c=1 if a==2; c=a-2 if a>3; }\\
\\
\\
Carmichael's theorem:\\
if gcd(a,n)==1\\
    then $\a^{\lambda (n)} \equiv 1 \pmod n$

\subsubsection{Prime}
Prime number theorem:\\
Let $\pi$(x) be the prime-counting function that gives the number of primes less than or equal to x, for any real number x.\\
$\lim\limits_{x\to\infty}\frac{\pi(x)}{x/\ln(x)}=1$\\
known as the asymptotic law of distribution of prime numbers.\\
$\pi(x)\sim\frac{x}{\ln x}.\!$\\

\begin{lstlisting}[language=C++]
#include<vector>

std::vector<int>prm;
bool flag[MAXX];

int main()
{
    prm.reserve(MAXX); // pi(x)=x/ln(x);
    for(i=2;i<MAXX;++i)
    {
        if(!flag[i])
            prm.push_back(i);
        for(j=0;j<prm.size() && i*prm[j]<MAXX;++j)
        {
            flag[i*prm[j]]=true;
            if(i%pmr[j]==0)
                break;
        }
    }
    return 0;
}
\end{lstlisting}
\subsubsection{M\"{o}bius Fuction}
\begin{itemize}
\item $\mu$(n) = 1 if n is a square-free positive integer with an even number of prime factors.
\item $\mu$(n) = −1 if n is a square-free positive integer with an odd number of prime factors.
\item $\mu$(n) = 0 if n has a squared prime factor.
\end{itemize}
$\sum_{d | n} \mu(d) = \begin{cases}1&\mbox{ if } n=1\\
        0&\mbox{ if } n>1.\end{cases}$
\begin{lstlisting}[language=C++]
int fac[MAXX],mob[MAXX];
bool flag[MAXX];
std::vector<int>prm(MAXX);
void go()
{
    static int i,j,k;
    prm.resize(0);
    mob[1]=1;
    for(i=2;i<MAXX;++i)
    {
        if(!flag[i])
        {
            prm.push_back(i);
            mob[i]=-1;
            fac[i]=1;
        }
        for(j=0;j<prm.size() && (k=i*prm[j])<MAXX;++j)
        {
            flag[k]=true;
            fac[k]=fac[i]+1;
            if(i%prm[j]==0)
            {
                mob[k]=0;
                break;
            }
            mob[k]=-mob[i];
        }
    }
}
inline ll cal(int n,int m)// [1,n] [1,m]`之间互质的数的对数`
{
    static int i,j,k;
    static ll re;
    re=0;
    for(i=1;i<=n;i=j+1)
    {
        j=std::min(n/(n/i),m/(m/i));
        re+=(mob[j]-mob[i-1])*(n/i)*(ll)(m/i); //`mob表示前缀和`
    }
    return re;
}
\end{lstlisting}

\subsubsection{Euler–Mascheroni constant}
$\gamma = \lim\limits_{n \rightarrow \infty } \left( \sum\limits_{k=1}^n \frac{1}{k} - \ln(n) \right)=\int\limits_1^\infty\left({1\over\lfloor x\rfloor}-{1\over x}\right)\,dx$\\
0.57721566490153286060651209008240243104215933593992...\\
\subsubsection{Fibonacci}
gcd(fib[i],fib[j])=fib[gcd(i,j)]\\
$a_{n}=\frac{\sqrt{5}}{5} \cdot \left[\left(\frac{1 + \sqrt{5}}{2}\right)^{n} - \left(\frac{1 - \sqrt{5}}{2}\right)^{n}\right]$
\subsubsection{Hensel's lemma}
设f(x)为有整系数多项式,k为不少于2的整数,p为质数。若整数r满足:\\
$f(r) \equiv 0 \pmod{p^{k-1}}$\\
对于$f(r+tp^{k-1}) \equiv 0 \pmod{p^k}$\\
则有
\begin{itemize}
\item 若$f'(r)\not\equiv 0\pmod{p}$,则存在唯一的整数$0\le t\le{p-1}$使上式成立
\item 若$f'(r)\equiv 0\pmod{p}$,且$f(r)\equiv 0\pmod{p^k}$,则任意正整数t都成立
\item 若$f'(r)\equiv 0\pmod{p}$,但$f(r)\not\equiv 0\pmod{p^k}$,则t不存在整数解
\end{itemize}
\subsection{Probability theory}
\subsubsection{Dice}
\begin{itemize}
\item m面体骰子连续n次掷出同一数字的投掷次数期望: $\frac{m^n-1}{m-1}$
\item m面体骰子连续n次掷出不同数字的投掷次数期望:\\
$ans=\sum d_i$\\
$d_{i+1}=d_i\times \frac{m}{m-i}, d_1=1$
\end{itemize}

%\subsubsection{Probability distribution}
%\paragraph{Binomial distribution}
%binomial distribution is the discrete probability distribution of the number of successes in a sequence of n independent yes/no experiments, each of which yields success with probability p.

